\documentclass[a4paper, 11pt]{article}
\usepackage{jipkg}
\usepackage{tikz-uml}
\usepackage{pdflscape}

%Titre et auteur
\jihypersetup{sta UML}{Quentin Ribac}
\title{sta UML}
\author{Quentin \textsc{Ribac}}
\date{\today}

\begin{document}
%Début du document
\selectlanguage{english}
\maketitle
\section*{Introduction}
This document is meant to present the different UML diagrams which will be made in the process of designing \emph{sta}.

UML diagrams in this document are set in \LaTeX{} using the \texttt{tikz-uml} package made by Nicolas \textsc{Kielbasiewicz}. See \url{http://ensta-paristech.fr/~kielbasi/tikz-uml} for more information this package. I am also using my own package \texttt{jipkg} for usual inclusions. See it on GitHub at \url{https://github.com/ribacq/jiltx}.

\section{Class Diagram for the \texttt{server} package}
\begin{figure}[h]\label{pkg: server}
\begin{tikzpicture}
	\begin{umlpackage}[x = 0, y = 0]{server}
		% Room, Character & Item
		\umlsimpleclass[x = 5, y = 0]{Room}
		\umlassoc[mult1 = *, mult2 = *, angle1 = 180, angle2 = -135, loopsize = 2cm]{Room}{Room}
		\umlsimpleclass[x = 5, y = -10]{Character}
		\umluniassoc[stereo = is in, mult1 = *, mult2 = 1, pos1 = 0.1, pos2 = 0.9]{Character}{Room}
		\umlsimpleclass[x = 8, y = -5]{Item}

		% Game
		\umlsimpleclass[x = 0, y = -5]{Game}
		\umlassoc[mult1 = 1, mult2 = 0..1]{Character}{Game}

		% ItemContainer
		\umlsimpleinterface[x = 12, y = -5]{ItemContainer}
		\umlimpl[angle1 = 0]{Room}{ItemContainer}
		\umlimpl[angle1 = 0]{Character}{ItemContainer}
		\umlimpl[anchor1 = 10, anchor2 = 170]{Item}{ItemContainer}
		\umluniassoc[stereo = is in, mult1 = 0..*, mult2 = 1, anchor1 = -10, anchor2 = -170]{Item}{ItemContainer}

		% Event
		\umlsimpleclass[x = 8, y = 0]{Event}
		\umlsimpleclass[x = 12, y = 0]{Trigger}
		\umlsimpleclass[x = 12, y = -1.5]{Action}
		\umlassoc[mult1 = *, mult2 = 1..*]{Event}{Trigger}
		\umlassoc[mult1 = *, mult2 = 1]{Event}{Action}
	\end{umlpackage}
\end{tikzpicture}
\caption{Package: \texttt{server}}
\end{figure}

\section{Interfaces}
\begin{figure}[h]\label{itf: ItemContainer}\center{
\begin{tikzpicture}
	\umlinterface{ItemContainer}{
		− limit : int\\
		− contents : Item[]
	}{
		+ push(Item) : void\\
		+ pop() : Item
	}
\end{tikzpicture}
\caption{Interface: \texttt{ItemContainer}}
}\end{figure}

\begin{figure}[h]\label{itf: Lookable}\center{
\begin{tikzpicture}
	\umlinterface{Lookable}{
		− name : string\\
		− desc : string
	}{
		+ look() : string
	}
\end{tikzpicture}
\caption{Interface: \texttt{Lookable}}
}\end{figure}

\section{Detailed classes}
\subsection{Game elements}
\begin{figure}[h]\label{cls: Room}\center{
\begin{tikzpicture}
	\umlclass{Room}{
		%
	}{
		%
	}
	\umlassoc[mult1 = *, mult2 = *]{Room}{Room}
\end{tikzpicture}
\caption{Class: \texttt{Room}}
}\end{figure}

\begin{figure}[h]\label{cls: Item}\center{
\begin{tikzpicture}
	\umlclass{Item}{
		%
	}{
		%
	}
\end{tikzpicture}
\caption{Class: \texttt{Item}}
}\end{figure}

\begin{figure}[h]\label{cls: Character}\center{
\begin{tikzpicture}
	\umlclass{Character}{
		%
	}{
		%
	}
\end{tikzpicture}
\caption{Class: \texttt{Character}}
}\end{figure}

\listoffigures

\end{document}

